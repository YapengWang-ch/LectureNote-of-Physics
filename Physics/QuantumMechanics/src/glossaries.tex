\newglossaryentry{picture}{
    name = {绘景},
    description = {绘景(picture)是量子力学中描述量子态随时间演化的一种方法。常见的绘景有薛定谔绘景、海森堡绘景和相互作用绘景。在薛定谔绘景中,量子态随时间变化,而算符保持不变;在海森堡绘景中,量子态保持不变,而算符随时间变化;在相互作用绘景中,量子态和算符都随时间变化,但变化方式不同。选择不同的绘景可以简化问题的求解过程。},
    first = {绘景(picture)}
}
\newglossaryentry{commutator}{
    name = {对易子},
    description = {对易子(commutator)是量子力学中用于描述两个算符之间关系的数学工具。对于两个算符 \(A\) 和 \(B\),它们的对易子定义为 \([A, B] = AB - BA\)。如果对易子为零,即 \([A, B] = 0\),则称这两个算符对易,意味着它们可以同时具有确定的测量值。对易子在量子力学中起着重要作用,特别是在描述物理量的测量和不确定性原理时。},
    first = {对易子(commutator)}
}
\newglossaryentry{eigenstate}{
    name = {本征态},
    description = {本征态(eigenstate)是量子力学中描述量子系统状态的一种特殊状态。当一个量子态是某个算符的本征态时,测量该算符对应的物理量会得到一个确定的值,称为本征值。数学上,如果 \(A\) 是一个算符,\(\psi\) 是它的本征态,那么满足方程 \(A\psi = a\psi\),其中 \(a\) 是对应的本征值。本征态在量子力学中具有重要意义,因为它们描述了系统在测量时可能的状态。},
    first = {本征态(eigenstate)}
}

\newglossaryentry{QM}{
    name = {量子力学},
    description = {量子力学(Quantum Mechanics)是研究微观粒子行为和性质的物理学分支,是现代物理学体系的重要基石。},
    first = {量子力学(Quantum Mechanics)}
}

\newglossaryentry{bra}{
    name = {左矢},
    description = {在量子力学中,左矢(bra)是希尔伯特空间中的一个元素,通常表示为 \(\langle \psi |\),它是一个线性函数,可以作用于右矢(ket)以产生一个复数。左矢与右矢一起构成了内积的基础,用于描述量子态之间的关系和测量结果。},
    first = {左矢(bra)}
}

\newglossaryentry{ket}{
    name = {右矢},
    description = {在量子力学中,右矢(ket)是希尔伯特空间中的一个元素,通常表示为 \(| \psi \rangle\),它描述了量子系统的状态。右矢可以与左矢(bra)结合形成内积,用于计算量子态之间的关系和测量结果。},
    first = {右矢(ket)}
}

\newglossaryentry{operator}{
    name = {算符},
    description = {在量子力学中,算符(operator)是作用在希尔伯特空间中的线性映射,用于描述物理量的测量和量子态的演化。常见的算符包括位置算符、动量算符和哈密顿算符等。算符通常表示为大写字母,如 \(A\)、\(B\) 等,并且可以通过对易关系来描述它们之间的相互作用。},
    first = {算符(operator)}
}

\newglossaryentry{HilbertSpace}{
    name = {希尔伯特空间},
    description = {希尔伯特空间(Hilbert Space)是量子力学中用于描述量子态的数学结构。它是一个完备的内积空间,允许定义向量的长度和角度,从而可以进行正交化和归一化等操作。希尔伯特空间中的每个向量对应一个量子态,而线性算符则作用于这些向量以描述物理量的测量和系统的演化。},
    first = {希尔伯特空间(Hilbert Space)}
}

\newglossaryentry{SPicture}{
    name={薛定谔绘景},
    description={Schrodinger's Picture},
    first={薛定谔绘景(Schrodinger's Picture)}
}

\newglossaryentry{HPicture}{
    name={海森堡绘景},
    description={Heisenberg's Picture},
    first={海森堡绘景(Heisenberg's Picture)}
}

\newglossaryentry{harmonic_sphere}{
    name = {球谐函数},
    description={.}
}