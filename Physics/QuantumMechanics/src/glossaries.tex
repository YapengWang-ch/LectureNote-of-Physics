\newglossaryentry{picture}{
    name = {绘景},
    description = {绘景(picture)是量子力学中描述量子态随时间演化的一种方法。常见的绘景有薛定谔绘景、海森堡绘景和相互作用绘景。在薛定谔绘景中,量子态随时间变化,而算符保持不变;在海森堡绘景中,量子态保持不变,而算符随时间变化;在相互作用绘景中,量子态和算符都随时间变化,但变化方式不同。选择不同的绘景可以简化问题的求解过程。},
    first = {绘景(picture)}
}
\newglossaryentry{commutator}{
    name = {对易子},
    description = {对易子(commutator)是量子力学中用于描述两个算符之间关系的数学工具。对于两个算符 \(A\) 和 \(B\),它们的对易子定义为 \([A, B] = AB - BA\)。如果对易子为零,即 \([A, B] = 0\),则称这两个算符对易,意味着它们可以同时具有确定的测量值。对易子在量子力学中起着重要作用,特别是在描述物理量的测量和不确定性原理时。},
    first = {对易子(commutator)}
}
\newglossaryentry{eigenstate}{
    name = {本征态},
    description = {本征态(eigenstate)是量子力学中描述量子系统状态的一种特殊状态。当一个量子态是某个算符的本征态时,测量该算符对应的物理量会得到一个确定的值,称为本征值。数学上,如果 \(A\) 是一个算符,\(\psi\) 是它的本征态,那么满足方程 \(A\psi = a\psi\),其中 \(a\) 是对应的本征值。本征态在量子力学中具有重要意义,因为它们描述了系统在测量时可能的状态。},
    first = {本征态(eigenstate)}
}