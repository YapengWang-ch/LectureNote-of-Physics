\chapter{Angular Momentum and Rotation}
\section{Angular Momentum}
定义角动量算符\begin{equation}
    \op{L}=\op{x}\times\op{p}.
\end{equation}

在平面直角坐标中有三个分量:
\begin{align}
    \op{L}_x&=\op{y}\op{p}_z-\op{z}\op{p}_y\\
    \op{L}_y&=\op{z}\op{p}_x-\op{x}\op{p}_z\\
    \op{L}_z&=\op{x}\op{p}_y-\op{y}\op{p}_x\\
    \op{L}^2&=\op{L}_x^2+\op{L}_y^2+\op{L}_z^2
\end{align}

容易证明,角动量算符的对易关系为:
\begin{align}
    \com{\op{L}_x}{\op{L}_x}&=0;\\
    \com{\op{L}_x}{\op{L}_y}&=i\hbar\op{L}_z;\\
    \com{\op{L}_x}{\op{L}_z}&=-i\hbar\op{L}_y.
\end{align}
其余分量可以此类推。

但是角动量的平方与各个分量都是对易的:
\begin{equation}
    \com{\op{L}^2}{\op{L}_i}=0, \quad i=x,y,z.
\end{equation}

由于角动量的三个分量并不对易,这意味着量子系统中只能同时准确观测到总角动量的幅值和某一个方向的角动量分量。我们取$\op{L}^2$和$\op{L}_z$作为描述角动量的一组力学量完全集。我们可以找到一组量子态$\ket{\psi}$,满足:
\begin{equation}
    \op{L}^2\ket{\psi}=\lambda\ket{\psi},\quad \op{L}_z\ket{\psi}=\mu\ket{\psi}.
\end{equation}

\section{the Quantum Number of Angular Momentum}

仿照谐振子中的算符,可以定义角动量的升降算符:
\begin{equation}
    \op{L}_\pm\equiv \op{L}_x\pm\op{L}_y.
\end{equation}

易证,升降算符的对易关系为:
\begin{equation}
    \com{\op{L}^2}{\op{L}_\pm}=0, \quad \com{\op{L}_z}{\op{L}_\pm}=\pm\hbar\op{L}_\pm.
\end{equation}
可以证明,对于$\op{L}^2,\op{L}_z$的共同本征态$\ket{\psi}$,$\op{L}_\pm\ket{\psi}$也是$\op{L}^2,\op{L}_z$的共同本征态.

\begin{graybox}[Proof]
    对于$\op{L}_\pm\ket{\psi}$:

    \begin{equation}
        \begin{aligned}
            \op{L}^2\op{L}_\pm\ket{\psi}=\op{L}_\pm\op{L}^2\ket{\psi}=\lambda\op{L}_\pm\ket{\psi}.
        \end{aligned}
    \end{equation}

    \begin{equation}
        \begin{aligned}
            \op{L}_z\op{L}_\pm\ket{\psi}&=\op{L}_\pm\op{L}_z\ket{\psi}\pm\hbar\op{L}_\pm\ket{\psi}\\
            &=(\mu\pm\hbar)\op{L}_\pm\ket{\psi}.
        \end{aligned}
    \end{equation}
\end{graybox}

与简谐振子相同,$\op{L}_z$的本征值也不可能无限增高或降低,存在两个本征态满足:
\begin{equation}
    \op{L}_+\ket{\psi}_\text{highest}=0,\quad \op{L}_-\ket{\psi}_\text{lowest}=0.
\end{equation}
假设$$\op{L}^z\ket{\psi}_\text{highest} =\lambda_{\max}\ket{\psi}_\text{highest},\quad \op{L}^z\ket{\psi}_\text{lowest} =\lambda_{\min}\ket{\psi}_\text{lowest}$$

可以将$\op{L}^2$写成:
\begin{equation}
    \op{L}^2=\op{L}_\pm\op{L}_\mp+\op{L}_z^2\mp\hbar\op{L}_z.
\end{equation}

那么
\begin{equation}
    \begin{aligned}
        \op{L}^2\ket{\psi}_\text{highest}&=\op{L}_z^2\mp\hbar\op{L}_z\ket{\psi}_\text{highest}\\
        &=\mu^2+\mu\hbar\ket{\psi}_\text{highest}\\
        &=\mu(\mu+\hbar)\ket{\psi}_\text{highest}.
    \end{aligned}
\end{equation}
即$\mu$最大满足$\mu_\max(\mu_\max+\hbar)=\lambda$,同理$\mu$最小满足$\mu_\min(\mu_\min-\hbar)=\lambda$.

这说明$\mu_\min=-\mu_\max$,且$\mu_\max-\mu_\min$为$\hbar$的整数倍。取量子数$m$标记$\op{L}_z$的本征值:
\begin{equation}
    \mu=m\hbar,\quad m=-l,-l+1,\cdots,l-1,l;\quad l=0,1/2,1,3/2,\cdots.
\end{equation}
显然$m$只能为整数或半整数。

对应的\begin{equation}
    \lambda=l(l+1)\hbar^2.
\end{equation}

\section{Rotation}



