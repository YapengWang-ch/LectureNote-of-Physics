\chapter{Quantum Dynamics}
\section{Time Evolution}
量子态的时间演化由薛定谔方程描述:
\begin{equation}
    i\hbar \frac{\partial}{\partial t} \ket{\psi(t)} = \op{H} \ket{\psi(t)}
\end{equation}
其中 $\op{H}$ 是系统的哈密顿算符:
\begin{equation}
    \op{H}=-\frac{\op{P}^2}{2m}+V(x)
\end{equation}

\section{Schrodinger Picture and Heisenberg Picture}
对于量子系统的演化,有两种不同的视角,分别称为\gls{SPicture}和\gls{HPicture}。

在薛定谔绘景中,物理量代表的算符是不变的,量子态随时间变化。而在海森堡绘景中,量子态是不变的,物理量代表的算符随时间变化。这两种不同的绘景在物理意义上是等价的。
\section{partile in one-dimensional potential}
任何物理态都可以视为若干个本征态的线性叠加,在这里我们取能量本征态作为基矢。

对于稳定一维势场中的能量本征态$\ket{E_i}$,
\begin{equation}
    \op{U}(t)\ket{E_i}=\exp\left(-\frac{i\op{H}t}{\hbar}\right)\ket{E_i}=\exp\left(-\frac{iE_it}{\hbar}\right)\ket{E_i}.
\end{equation}
因此有
\begin{equation}
    \bra{x}\op{U}(t)\ket{E_i}=\exp\left(-\frac{iE_it}{\hbar}\right)\braket{x}{E_i}.
\end{equation}
可以在坐标表象进行定态求解。

$\forall \ket{x}, $有
\begin{equation}
    \bra{x}\op{H}\ket{E_i}=E_i\braket{x}{E_i}
\end{equation}
\begin{equation}
    \left[-\frac{\hbar^2}{2m}\ddto{}{x}+V(x)\right]\braket{x}{E_i}=E_i\braket{x}{E_i}
\end{equation}
自然地,其边界条件为:
\begin{enumerate}
    \item 对于$V(x)\neq \infty$的情形,$\frac{\hbar^2}{2m}\braket{x}{E_i}$存在且有限;
    \item 对于$V(x)=\pm\infty$的情形,$\frac{\hbar^2}{2m}\braket{x}{E_i}=\mp\infty$.
\end{enumerate}
\subsection{free particle}
对于自由粒子,其哈密顿量为
\begin{equation}
    \op{H}=\frac{\op{P}^2}{2m}
\end{equation}
显然,其动量本征态就是能量本征态,为$\ket{p}$, 在位置表象下为
\begin{equation}
    \psi(x,p)=\braket{x}{p}=\frac{1}{\sqrt{2\pi\hbar}} e^{ipx/\hbar}
\end{equation}
\subsection{infinite-deep potential well}
对于一维的有限深方势阱$V(x)$:
\begin{equation}
    V(x)=\left\lbrace\begin{aligned}
        &=-V_0, && x\in[0,a];\\
        &=0, && x <0 \ \mathtt{or}\  x>a.
    \end{aligned}\right.
\end{equation}
记$\braket{x}{E_i}=\psi(x)$分为三段,
\begin{equation}
    \psi(x)=\left\lbrace\begin{aligned}
        &\psi_1(x),&&x<0;\\
        &\psi_2(x),&&0\leq x\leq a;\\
        &\psi_3(x),&&x>a.
    \end{aligned}\right.
\end{equation}
其边界条件为: 
\begin{align}
    \psi_1(0)&=\psi_2(0)\\
    \psi_2(a)&=\psi_3(a)\\
    \psi_1'(0)&=\psi_2'(0)\\
    \psi_2'(a)&=\psi_3'(a)
\end{align}
\subsection{finite potential well}
\subsection{potential barrier \& tunneling effect}
\subsection{delta potential well \& barrier}
对于$\delta$势阱
\begin{equation}
    V(x)=-V_0\delta(x),
\end{equation}
波函数分为束缚态$E\le 0$和散射态$E>0$.

\paragraph{散射态} 对于从左侧入射的波函数:
\begin{align}
    \psi_1(x)&=\exp\left(i\frac{px}{\hbar}\right)+R\exp\left(-i\frac{px}{\hbar}\right),x<0;\\
    \psi_2(x)&=T\exp\left(i\frac{px}{\hbar}\right),  x>0.
\end{align}
在$x=0$处进行势函数积分:
\begin{equation}
    \int_{-\varepsilon}^{+\varepsilon}\left[-\frac{\hbar^2}{2m}\psi''(x)+V(x)\psi(x)-E\psi(x)\right]\d x=0.
\end{equation}
积分得:
\begin{equation}
    -\frac{\hbar^2}{2m}(\psi'(x)|_{0^+}-\psi'(x)|_{0^-})-V_0\psi(0)=0.
    \label{delta_boundary}
\end{equation}
结合边界条件$\psi(0)|_{0^+}=\psi(0)|_{0^-}$,可以解得:
\begin{equation}
    T=\frac{1}{1+\beta}, R=\frac{-\beta}{1+\beta}, \beta=\frac{mV_0}{ip\hbar}.
\end{equation}

\paragraph{束缚态} 双边波函数为指数形式,边界条件同式\ref{delta_boundary}.有唯一解:
\begin{equation}
    E=-\frac{mV_0^2}{2\hbar^2}.
\end{equation}

\paragraph{$\delta$势垒}与之相似,$\delta$势垒仅有散射态,其解为:
\begin{equation}
    T=\frac{1}{1-\beta}, R=\frac{\beta}{1-\beta}, \beta=\frac{mV_0}{ip\hbar}.
\end{equation}
\section{Simple Harmonic Oscillator}
对于一维简谐势场
\begin{equation}
    V(x)=\frac{1}{2}m\omega^2x^2.
\end{equation}
\subsection{the solution in position presentation}
在坐标表象中,定态薛定谔方程为:
\begin{equation}
    -\frac{\hbar^2}{2m}\ddto{\psi}{x}+\frac{1}{2}m\omega^2x^2\psi=E\psi.
\end{equation}
引入无量纲量$\xi=\sqrt{\frac{m\omega}{\hbar}}x=\alpha x$,将方程简化为:
\begin{equation}
    \ddto{\psi}{\xi}+(\lambda-\xi^2)\psi=0, \quad \lambda=\frac{2E}{\hbar\omega}.
\end{equation}

尝试渐进求解,当$\xi\to\infty$时,忽略$\lambda$, 方程为:
\begin{equation}
    \ddto{\psi}{\xi}=\xi^2\psi.
\end{equation}
其解$\psi\sim e^{\pm\xi^2/2}$.

因此,将原方程的解$\psi$写成$\psi=H(\xi)e^{-\frac{\xi^2}{2}}$,其中$H(\xi)$可通过多项式法进行求解。

将$H(x)$带入可以得到:
\begin{equation}
    \ddto{H}{\xi}-2\xi\dd{H}{\xi}+(\lambda-1)H=0.
\end{equation}
这个方法得到的解称为厄米多项式(详见附录\ref{Hermition_Polynomial}),其存在收敛解的条件为
\begin{equation}
    \lambda=2n+1, \quad n=0,1,2,3,\cdots.
\end{equation}

带入原方程可得波函数的本征值为:
\begin{equation}
    E_n=\left(n+\frac{1}{2}\right)\hbar\omega,\ n=0,1,2,3,\cdots. 
\end{equation}

\subsection{solution in quantum number presentation}

简谐势场的薛定谔方程还可以写成以下形式:
\begin{equation}
    \frac{1}{2m}\left(\op{p}^2+m^2\omega^2\op{x}^2\right)\psi=E\psi
\end{equation}
参考平方差公式,可以定义算符:
\begin{equation}
    a_{\pm}\equiv \frac{1}{\sqrt{2\hbar m\omega}}\left(\mp i\op{p}+m\omega \op{x}\right).
\end{equation}

但是$\op{x}$与$\op{p}$并不对易,$a_{\pm}$同样不对易,不能简单套用平方差公式,需要进行检验:
\begin{equation}
    \begin{aligned}
        a_-a_+&=\frac{1}{2\hbar m\omega}\left[\op{p}^2+\left(m\omega\op{x}\right)^2-im\omega\com{\op{x}}{\op{p}}\right]\\
        &=\frac{1}{\hbar\omega}\op{H}+\frac{1}{2}.
    \end{aligned}
\end{equation}
同理,有:
\begin{equation}
    a_+a_-=\frac{1}{\hbar\omega}\op{H}-\frac{1}{2}.
\end{equation}
\begin{equation}
    \op{H}=\hbar\omega\left(a_+a_-+\frac{1}{2}\right).
\end{equation}
可以得到这一对算符的对易关系:
\begin{equation}
    \com{a_-}{a_+}=1.
\end{equation}

容易证明,对于哈密顿量$\op{H}$的本征态$\psi$,$a_+\psi$也是$\op{H}$的本征态:
\begin{graybox}[Proof]
对于$\op{H}$和$a_+$:
\begin{equation}
    \begin{aligned}
        \com{\op{H}}{a_+}&=\hbar\omega\left(a_+a_-+\frac{1}{2}\right)a_+-a_+\hbar\omega\left(a_+a_-+\frac{1}{2}\right)\\
        &=\hbar\omega\left(a_+a_-a_+-a_+a_+a_-\right)\\
        &=\hbar\omega a_+\com{a_-}{a_+}\\
        &=\hbar\omega a_+.
    \end{aligned}
\end{equation}
同理
\begin{equation}
    \com{\op{H}}{a_-}=\hbar\omega a_-.
\end{equation}
那么对于$\op{H}$的本征态$\psi$:
\begin{equation}
    \op{H}\psi=E\psi.
\end{equation}
有 
\begin{equation}
    \op{H}a_+\psi=(E+\hbar\omega)a_+\psi.
\end{equation}
\begin{equation}
    \op{H}a_-\psi=(E-\hbar\omega)a_-\psi.
\end{equation}
\end{graybox}

但是谐振子的能量不可能为无限低,因此存在一个能量的下界使得:
\begin{equation}
    a_-\psi_0=0.
\end{equation}
将方程展开为:
\begin{equation}
    \begin{aligned}
        \frac{1}{\sqrt{2\hbar m\omega}}\left(i\op{p}+m\omega \op{x}\right)\psi_0&=0\\
        (-\hbar\pp{}{x}+m\omega x)\psi_0(x)&=0.
    \end{aligned}
\end{equation}
可以解得:
\begin{equation}
    \psi_0(x)=A\exp\left(-\frac{m\omega x^2}{2\hbar}\right)
\end{equation}
归一化为:
\begin{equation}
    \psi_0(x)=\sqrt{\frac{m\omega}{2\pi\hbar}}\exp\left(-\frac{m\omega x^2}{2\hbar}\right).
\end{equation}

\section{central field}
对于中心力场$V(\vec{x})=V(r)$,其定态薛定谔方程可写为:
\begin{equation}
    -\frac{\hbar^2}{2m}\nabla^2\psi(\vec{r})+(V(r)-E)\psi(\vec{r})=0
\end{equation}
其中Laplace算符$\nabla^2=\ppto{}{x}+\ppto{}{y}+\ppto{}{z}$可以按照极坐标展开为:
\begin{equation}
    \nabla^2=\frac{1}{r^2}\pp{}{r}\left(r^2\pp{}{r}\right)+\frac{1}{r^2}\left\lbrace \frac{1}{\sin\theta}\pp{}{\theta}\left(\sin\theta\pp{}{\theta}\right)+\frac{1}{\sin^2\theta}\ppto{}{\phi}\right\rbrace
\end{equation}
将$\psi(\vec{r})$写成$\frac{u(r)}{r}Y(\theta,\phi)$,则有:
\begin{equation}
    \nabla^2\psi(\vec{r})=\frac{u''(r)}{r}Y(\theta,\phi)+\frac{u(r)}{r^3}\left[ \frac{1}{\sin\theta}\pp{}{\theta}\left(\sin\theta\pp{}{\theta}\right)+\frac{1}{\sin^2\theta}\ppto{}{\phi}\right]Y(\theta,\phi).
\end{equation}
将径向部分$u(r)$与轴向部分$Y(\theta,\phi)$分离开:
\begin{align}
    \left[ \frac{1}{\sin\theta}\pp{}{\theta}\left(\sin\theta\pp{}{\theta}\right)+\frac{1}{\sin^2\theta}\ppto{}{\phi}\right]Y(\theta,\phi)=&AY(\theta,\phi)\\
    -\frac{\hbar^2}{2m}u''(r)-\frac{A\hbar^2}{2m}\frac{u(r)}{r^2}+(V(r)-E)u(r)=&0
\end{align}
其中轴向部分的解为\gls{harmonic_sphere}(详见附录):
\begin{equation}
    \left[ \frac{1}{\sin\theta}\pp{}{\theta}\left(\sin\theta\pp{}{\theta}\right)+\frac{1}{\sin^2\theta}\ppto{}{\phi}\right]Y_{l,m}(\theta,\phi)=l(l+1)Y_{l,m}(\theta,\phi)
\end{equation}
径向部分需要根据势能函数进行求解。

\subsection{Coulomb Field of Hydrogen Atom}

\newpage 
\section*{Problems}
\begin{enumerate}
    \item \begin{equation}
        -\frac{\hbar^2}{2m}\ppto{}{x}\psi(x)=(E-ax)\psi(x)
    \end{equation}
    \item 对于中心势场
    \begin{equation}
        V(r)=\left\lbrace
        \begin{aligned}
            &-V_0, && r<a;\\
            &0, && r\ge a.
        \end{aligned}\right.
    \end{equation}
    求$V_0$的最小值,使得能量$E=0$的无角动量态存在。
\end{enumerate}

\newpage
\section*{Solution to Problems}
\begin{enumerate}
    \item a
    \item 无角动量,即$l=0$, 径向部分为:
        \begin{equation}
                -\frac{\hbar^2}{2m}u''(r)+(V(r)-E)u(r)=0.
        \end{equation}
        可以写成:
        \begin{equation}
            u''(r)-\alpha u(r)=0,\quad \alpha = \frac{2m}{\hbar^2}(V-E).
        \end{equation}
        对于$r<a$, $\alpha=-\frac{2mV_0}{\hbar^2}<0$, 其解为 
        \begin{equation}
            u_1(r)=A\sin\left(\sqrt{-\frac{2mV_0}{\hbar^2}}r\right),r<a.
        \end{equation}
        对于$r\ge a$, $\alpha=0$, 其解为 
        \begin{equation}
            u_2(r)=Br+C, r\ge a
        \end{equation}
        边界条件为:
        \begin{align}
            u_1(a)&=u_2(a) &&\to &A\sin\left(\sqrt{\frac{2mV_0}{\hbar^2}}a\right)&=Ba+C\\
            u_1'(a)&=u'_2(a) &&\to &\sqrt{\frac{2mV_0}{\hbar^2}}A\cos\left(\sqrt{\frac{2mV_0}{\hbar^2}}a\right)&=B\\
            \lim_{r\to+\infty}\frac{u_2(r)}{r}&=0 &&\to &B&=0.
        \end{align}
        解得\begin{equation}
            A\sin[(n+\frac{1}{2})\pi]=C, \quad\sqrt{\frac{2mV_0}{\hbar^2}}a=(n+\frac{1}{2})\pi.
        \end{equation}
        对应$V_0$最小为$\frac{\pi^2\hbar^2}{8ma^2}$.
\end{enumerate}


