\chapter{Quantum Dynamics}
\section{Time Evolution}
量子态的时间演化由薛定谔方程描述:
\begin{equation}
    i\hbar \frac{\partial}{\partial t} \ket{\psi(t)} = \op{H} \ket{\psi(t)}
\end{equation}
其中 $\op{H}$ 是系统的哈密顿算符:
\begin{equation}
    \op{H}=-\frac{\op{P}^2}{2m}+V(x)
\end{equation}

\section{Schrodinger Picture and Heisenberg Picture}
对于量子系统的演化,有两种不同的视角,分别称为\gls{SPicture}和\gls{HPicture}。

在薛定谔绘景中,物理量代表的算符是不变的,量子态随时间变化。而在海森堡绘景中,量子态是不变的,物理量代表的算符随时间变化。这两种不同的绘景在物理意义上是等价的。
\section{partile in one-dimensional potential}
任何物理态都可以视为若干个本征态的线性叠加,在这里我们取能量本征态作为基矢。

对于稳定一维势场中的能量本征态$\ket{E_i}$,
\begin{equation}
    \op{U}(t)\ket{E_i}=\exp\left(-\frac{i\op{H}t}{\hbar}\right)\ket{E_i}=\exp\left(-\frac{iE_it}{\hbar}\right)\ket{E_i}.
\end{equation}
因此有
\begin{equation}
    \bra{x}\op{U}(t)\ket{E_i}=\exp\left(-\frac{iE_it}{\hbar}\right)\braket{x}{E_i}.
\end{equation}
可以在坐标表象进行定态求解。

$\forall \ket{x}, $有
\begin{equation}
    \bra{x}\op{H}\ket{E_i}=E_i\braket{x}{E_i}
\end{equation}
\begin{equation}
    \left[-\frac{\hbar^2}{2m}+\ddto{}{x}+V(x)\right]\braket{x}{E_i}=E_i\braket{x}{E_i}
\end{equation}
自然地,其边界条件为:
\begin{enumerate}
    \item 对于$V(x)\neq \infty$的情形,$\frac{\hbar^2}{2m}\braket{x}{E_i}$存在且有限;
    \item 对于$V(x)=\pm\infty$的情形,$\frac{\hbar^2}{2m}\braket{x}{E_i}=\mp\infty$.
\end{enumerate}
\subsection{free particle}
对于自由粒子,其哈密顿量为
\begin{equation}
    \op{H}=\frac{\op{P}^2}{2m}
\end{equation}
显然,其动量本征态就是能量本征态,为$\ket{p}$, 在位置表象下为
\begin{equation}
    \psi(x,p)=\braket{x}{p}=\frac{1}{\sqrt{2\pi\hbar}} e^{ipx/\hbar}
\end{equation}
\subsection{infinite-deep potential well}
对于一维的有限深方势阱$V(x)$:
\begin{equation}
    V(x)=\left\lbrace\begin{aligned}
        &=-V_0, && x\in[0,a];\\
        &=0, && x <0 \ \mathtt{or}\  x>a.
    \end{aligned}\right.
\end{equation}
记$\braket{x}{E_i}=\psi(x)$分为三段,
\begin{equation}
    \psi(x)=\left\lbrace\begin{aligned}
        &\psi_1(x),&&x<0;\\
        &\psi_2(x),&&0\leq x\leq a;\\
        &\psi_3(x),&&x>a.
    \end{aligned}\right.
\end{equation}
其边界条件为: 
\begin{align}
    \psi_1(0)&=\psi_2(0)\\
    \psi_2(a)&=\psi_3(a)\\
    \psi_1'(0)&=\psi_2'(0)\\
    \psi_2'(a)&=\psi_3'(a)
\end{align}
\subsection{finite potential well}
\subsection{potential barrier \& tunneling effect}
\subsection{delta potential well \& barrier}
\section{harmonic oscillator}




