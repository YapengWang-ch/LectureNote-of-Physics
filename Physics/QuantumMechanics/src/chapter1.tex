\chapter{Fundamental Concepts}
% \section{Fundamental Postulates of Quantum Mechanics}
\gls{QM} 有以下几个基本假设:
% \begin{itemize}
%     \item 
% \end{itemize}
\begin{enumerate}
    \item 系统的状态由\gls{HilbertSpace}中的矢量描述。
    \item 可观测量由厄米算符表示,测量结果为该算符的本征值。
    \item 态矢量的演化由薛定谔方程描述。
\end{enumerate}

\section{Bra , Ket \& Operators}

在量子力学中,系统的状态由\gls{HilbertSpace}中的矢量表示,通常称为\gls{ket},记作 $\ket{\psi}$。

\gls{ket}的复共轭称为\gls{bra},记作 $\bra{\psi}$。

\gls{operator}是作用在\gls{HilbertSpace}上的线性映射,通常用花体字母表示,如 $\op{A}$。算符的作用是将一个\gls{ket}映射到另一个\gls{ket},即 $\op{A}\ket{\psi} = \ket{\phi}$。

\subsection{Inner Product \& Outer Product}
\gls{bra}与\gls{ket}可以相乘得到一个复数,称为内积,记作 $\braket{\phi}{\psi}$。 内积满足以下性质:
\begin{itemize}
    \item $\braket{\psi}{\psi} \geq 0$,且当且仅当 $\ket{\psi} = 0$ 时取等号。
    \item $\braket{\phi}{\psi} = \braket{\psi}{\phi}^*$
    \item $\braket{\phi}{(\alpha\ket{\psi_1} + \beta\ket{\psi_2})} = \alpha\braket{\phi}{\psi_1} + \beta\braket{\phi}{\psi_2}$
\end{itemize}

\gls{ket}与\gls{bra}相乘得到一个算符,称为外积,记作 $\ket{\psi}\bra{\phi}$。 外积的作用是将 $\ket{\phi}$ 映射到 $\ket{\psi}$,即 $(\ket{\psi}\bra{\phi})\ket{\phi} = \ket{\psi}$。

\subsection{Hermitian Operators}
算符 $\op{A}$ 的厄密共轭记作 $\op{A}^\dagger$,定义为满足以下关系的算符:
\begin{equation}
    \braket{\phi}{\op{A}\psi} = \braket{\op{A}^\dagger\phi}{\psi}
\end{equation}
如果 $\op{A} = \op{A}^\dagger$,则称 $\op{A}$ 为厄米算符。厄米算符具有以下性质:
\begin{itemize}
    \item 本征值为实数。
    \item 不同本征值对应的本征矢量正交。
    \item 可以构成完备归一化的本征矢量组。
\end{itemize}

在量子力学中,可观测量由厄米算符表示,测量结果为该算符的本征值。

\subsection{measurement}
