\chapter{Fundamental Concepts}
% \section{Fundamental Postulates of Quantum Mechanics}
\gls{QM} 有以下几个基本假设:
% \begin{itemize}
%     \item 
% \end{itemize}
\begin{enumerate}
    \item 系统的状态由\gls{HilbertSpace}中的矢量描述。
    \item 可观测量由厄米算符表示,测量结果为该算符的本征值。
    \item 态矢量的演化由薛定谔方程描述。
\end{enumerate}

\section{Bra , Ket \& Operators}

在量子力学中,系统的状态由\gls{HilbertSpace}中的矢量表示,通常称为\gls{ket},记作 $\ket{\psi}$。

\gls{ket}的复共轭称为\gls{bra},记作 $\bra{\psi}$。

\gls{operator}是作用在\gls{HilbertSpace}上的线性映射,通常用花体字母表示,如 $\op{A}$。算符的作用是将一个\gls{ket}映射到另一个\gls{ket},即 $\op{A}\ket{\psi} = \ket{\phi}$。

\subsection{Inner Product \& Outer Product}
\gls{bra}与\gls{ket}可以相乘得到一个复数,称为内积,记作 $\braket{\phi}{\psi}$。 内积满足以下性质:
\begin{itemize}
    \item $\braket{\psi}{\psi} \geq 0$,且当且仅当 $\ket{\psi} = 0$ 时取等号。
    \item $\braket{\phi}{\psi} = \braket{\psi}{\phi}^*$
    \item $\braket{\phi}{(\alpha\ket{\psi_1} + \beta\ket{\psi_2})} = \alpha\braket{\phi}{\psi_1} + \beta\braket{\phi}{\psi_2}$
\end{itemize}

\gls{ket}与\gls{bra}相乘得到一个算符,称为外积,记作 $\ket{\psi}\bra{\phi}$。 外积的作用是将 $\ket{\phi}$ 映射到 $\ket{\psi}$,即 $(\ket{\psi}\bra{\phi})\ket{\phi} = \ket{\psi}$。

\subsection{Hermitian Operators}
算符 $\op{A}$ 的厄密共轭记作 $\op{A}^\dagger$,定义为满足以下关系的算符:
\begin{equation}
    \braket{\phi}{\op{A}\psi} = \braket{\op{A}^\dagger\phi}{\psi}
\end{equation}
如果 $\op{A} = \op{A}^\dagger$,则称 $\op{A}$ 为厄米算符。厄米算符具有以下性质:
\begin{itemize}
    \item 本征值为实数。
    \item 不同本征值对应的本征矢量正交。
    \item 可以构成完备归一化的本征矢量组。
\end{itemize}

\subsection{Eigenvalues \& Eigenstates}
对于算符 $\op{A}$,如果存在非零矢量 $\ket{\psi}$ 和标量 $a$,使得
\begin{equation}
    \op{A}\ket{\psi} = a\ket{\psi}
\end{equation}
则称 $\ket{\psi}$ 为 $\op{A}$ 的本征态,$a$ 为对应的本征值。

在量子力学中,可观测量由厄米算符表示,测量结果为该算符的本征值。

\subsection{measurement}
测量一个可观测量 $\op{A}$ 时,系统的状态 $\ket{\psi}$ 会坍缩到 $\op{A}$ 的某个本征态 $\ket{a}$,测量结果为对应的本征值 $a$。测量结果 $a$ 出现的概率为
\begin{equation}
    P(a) = |\braket{a}{\psi}|^2
\end{equation}

测量后系统的状态变为 $\ket{a}$。

\subsection{commutation of operators \& uncertainty principle}
两个算符 $\op{A}$ 和 $\op{B}$ 的对易子定义为
\begin{equation}
    \com{\op{A}}{\op{B}} = \op{A}\op{B} - \op{B}\op{A}
\end{equation}
如果 $\com{\op{A}}{\op{B}} = 0$,则称 $\op{A}$ 和 $\op{B}$ 对易。对易的算符可以同时具有确定的测量值。

如果 $\com{\op{A}}{\op{B}} \neq 0$,则称 $\op{A}$ 和 $\op{B}$ 不对易。根据不确定性原理,两个不对易的可观测量不能同时具有确定的测量值。具体地,对于两个可观测量 $\op{A}$ 和 $\op{B}$,它们的测量结果的不确定性满足以下关系:
\begin{equation}
    \Delta A \Delta B \geq \frac{1}{2} |\expect{\com{\op{A}}{\op{B}}}|
\end{equation}
其中 $\Delta A$ 和 $\Delta B$ 分别表示测量结果的标准差,$\expect{\cdot}$ 表示期望值。

\section{Position \& Momentum Operators}
在一维空间中,位置算符 $\hat{x}$ 和动量算符 $\hat{p}$ 定义如下:
\begin{equation}
    \hat{x}\ket{x} = x\ket{x}
\end{equation}
\begin{equation}
    \hat{p}\ket{p} = p\ket{p}
\end{equation}
其中 $\ket{x}$ 和 $\ket{p}$ 分别为位置和动量的本征态。

位置本征态和动量本征态满足正交归一化条件:
\begin{equation}
    \braket{x'}{x} = \delta(x' - x)
\end{equation}
\begin{equation}
    \braket{p'}{p} = \delta(p' - p)
\end{equation}

其内积为
\begin{equation}
    \braket{x}{p} = \frac{1}{\sqrt{2\pi\hbar}} e^{ipx/\hbar}
\end{equation}

进而可以得到位置和动量算符的对易关系:
\begin{equation}
    \com{\hat{x}}{\hat{p}} = i\hbar
\end{equation}

\subsection{Representation in Position Space}

\subsection{Representation in Momentum Space}

\subsection{the translation operator}
平移算符 $\mathscr{T}(a)$ 定义为将位置平移 $a$ 的算符,即
\begin{equation}
    \mathscr{T}(a)\ket{x} = \ket{x + a}
\end{equation}