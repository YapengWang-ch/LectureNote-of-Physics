\documentclass[a4paper]{book}%article最高级标题为section,report为chapter
%\textit{%package area
\usepackage{graphicx}%插入图片
\usepackage{titlesec}%修改标题格式
\usepackage{amsmath}%大部分数学符号
\usepackage{mathrsfs}%花体
\usepackage{amssymb}%双线体mathbb
\usepackage{xeCJK}%中文支持
\usepackage{ctex} %中文字体
\usepackage{geometry}%几何绘图
\usepackage{setspace}%设置行距
\usepackage{multicol}%表格合并
\usepackage{multirow}%表格合并
\usepackage{subfigure}%次级图片
% \usepackage{hyperref}%调整标注与引用
\usepackage{indentfirst}%首行缩进}
\usepackage{tikz}%tikz绘图
\usetikzlibrary{calc}
\usepackage{listings}%代码插入
% \lstset{language=Matlab}
\usepackage{xcolor}
\usepackage{tcolorbox} % 用于创建自定义文本框
\usepackage[nottoc]{tocbibind} %目录中显示index,ref,glossary 等
\usepackage{longtable}%大型表格跨页显示
\usepackage{booktabs}
%\usepackage{mathpazo} %
\usepackage{pgfplots}

\usepackage{imakeidx}
\makeindex[columns=2, title=索引] 
\usepackage[title]{appendix}
% \renewcommand{\appendixname}{附录}
\usepackage[colorlinks=true, linkcolor=blue, citecolor=blue, urlcolor=blue]{hyperref} % 关键设置
% colorlinks=true 表示用颜色而非边框表示链接
% linkcolor=blue 设置内部链接(如目录、章节引用)为蓝色
% citecolor=blue 设置文献引用链接为蓝色
% urlcolor=blue 设置URL链接为蓝色
% hidelinks 用于去除链接周围的边框
\usepackage{glossaries}
\newglossaryentry{picture}{
    name = {绘景},
    description = {绘景(picture)是量子力学中描述量子态随时间演化的一种方法。常见的绘景有薛定谔绘景、海森堡绘景和相互作用绘景。在薛定谔绘景中,量子态随时间变化,而算符保持不变;在海森堡绘景中,量子态保持不变,而算符随时间变化;在相互作用绘景中,量子态和算符都随时间变化,但变化方式不同。选择不同的绘景可以简化问题的求解过程。},
    first = {绘景(picture)}
}
\newglossaryentry{commutator}{
    name = {对易子},
    description = {对易子(commutator)是量子力学中用于描述两个算符之间关系的数学工具。对于两个算符 \(A\) 和 \(B\),它们的对易子定义为 \([A, B] = AB - BA\)。如果对易子为零,即 \([A, B] = 0\),则称这两个算符对易,意味着它们可以同时具有确定的测量值。对易子在量子力学中起着重要作用,特别是在描述物理量的测量和不确定性原理时。},
    first = {对易子(commutator)}
}
\newglossaryentry{eigenstate}{
    name = {本征态},
    description = {本征态(eigenstate)是量子力学中描述量子系统状态的一种特殊状态。当一个量子态是某个算符的本征态时,测量该算符对应的物理量会得到一个确定的值,称为本征值。数学上,如果 \(A\) 是一个算符,\(\psi\) 是它的本征态,那么满足方程 \(A\psi = a\psi\),其中 \(a\) 是对应的本征值。本征态在量子力学中具有重要意义,因为它们描述了系统在测量时可能的状态。},
    first = {本征态(eigenstate)}
}

\newglossaryentry{QM}{
    name = {量子力学},
    description = {量子力学(Quantum Mechanics)是研究微观粒子行为和性质的物理学分支,是现代物理学体系的重要基石。},
    first = {量子力学(Quantum Mechanics)}
}

\newglossaryentry{bra}{
    name = {左矢},
    description = {在量子力学中,左矢(bra)是希尔伯特空间中的一个元素,通常表示为 \(\langle \psi |\),它是一个线性函数,可以作用于右矢(ket)以产生一个复数。左矢与右矢一起构成了内积的基础,用于描述量子态之间的关系和测量结果。},
    first = {左矢(bra)}
}

\newglossaryentry{ket}{
    name = {右矢},
    description = {在量子力学中,右矢(ket)是希尔伯特空间中的一个元素,通常表示为 \(| \psi \rangle\),它描述了量子系统的状态。右矢可以与左矢(bra)结合形成内积,用于计算量子态之间的关系和测量结果。},
    first = {右矢(ket)}
}

\newglossaryentry{operator}{
    name = {算符},
    description = {在量子力学中,算符(operator)是作用在希尔伯特空间中的线性映射,用于描述物理量的测量和量子态的演化。常见的算符包括位置算符、动量算符和哈密顿算符等。算符通常表示为大写字母,如 \(A\)、\(B\) 等,并且可以通过对易关系来描述它们之间的相互作用。},
    first = {算符(operator)}
}

\newglossaryentry{HilbertSpace}{
    name = {希尔伯特空间},
    description = {希尔伯特空间(Hilbert Space)是量子力学中用于描述量子态的数学结构。它是一个完备的内积空间,允许定义向量的长度和角度,从而可以进行正交化和归一化等操作。希尔伯特空间中的每个向量对应一个量子态,而线性算符则作用于这些向量以描述物理量的测量和系统的演化。},
    first = {希尔伯特空间(Hilbert Space)}
}
\makeglossaries

\usepackage{todonotes}
\usepackage{needspace}
\renewcommand{\figurename}{图}
\renewcommand{\tablename}{表}
\renewcommand{\glossaryname}{术语表}
%%三线表线距控制
\def\topline{\hline\hline\specialrule{0pt}{3pt}{0pt}}%双线,下间距2pt
\def\midline{\specialrule{0pt}{3pt}{0pt}\hline\specialrule{0pt}{3pt}{0pt}}%单线,下间距2pt	
\def\bottomline{\specialrule{0pt}{2pt}{0pt}\hline\hline}%双线,上间距2pt
%\renewcommand\arraystretch{1.2} %控制表格行距,置于table环境内
\renewcommand{\abstractname}{摘要}
\newcommand*{\num}{pi}

\newtheorem{theorem}{Theorem}[chapter]
\newtheorem{definition}{Definition}[chapter]
\newtheorem{lem}[theorem]{Lemma}

\newcommand{\op}[1]{\hat{#1}}%算符
\newcommand{\ket}[1]{\left|#1\right>}
\newcommand{\bra}[1]{\left<#1\right|}
\newcommand{\braket}[2]{\left<#1\middle|#2\right>}
\newcommand{\com}[2]{\left[#1,#2\right]}%对易子
\newcommand{\anticom}[2]{\left\lbrace#1,#2\right\rbrace}%反对易子
\newcommand{\expect}[1]{\overline{#1}}%期望,上划线形式
\newcommand{\expectket}[1]{\left<#1\right>}%算符的期望
\newcommand{\ampsq}[1]{\left|#1\right|^2}%幅值平方,复数平方
\newcommand{\half}{\frac{1}{2}}
\newcommand{\quat}{\frac{1}{4}}
\newcommand{\si}[1]{\,\mathrm{#1}} %单位格式
\newcommand{\der}{\mathrm{d}}				%导数符号
\newcommand{\dd}[2]{\frac{\der #1}{\der #2}}
\newcommand{\ddto}[2]{\frac{\der^2 #1}{\der #2^2}}
\newcommand{\dx}[1]{\dd{#1}{x}}
\newcommand{\dt}[1]{\dd{#1}{t}}
\newcommand{\p}{\partial}				%偏微分符号
\newcommand{\pp}[2]{\frac{\p #1}{\p #2}}
\newcommand{\ppto}[2]{\frac{\p^2 #1}{\p #2^2}}
\newcommand{\px}[1]{\pp{#1}{x}}
\newcommand{\pt}[1]{\pp{#1}{t}}
\renewcommand{\d}{\der}
\newcommand{\degree}{^\circ}

\renewcommand{\max}{\text{max}}
\renewcommand{\min}{\text{min}}

\tcbuselibrary{skins, breakable}
\definecolor{DarkGray}{RGB}{70,70,70}
\definecolor{LightGray}{RGB}{240,240,240}
\newtcolorbox{darkgraybox}[1][]{
    enhanced, % 使用 enhanced 引擎 :cite[10]
    fontupper=\large,
    colback=DarkGray,
    colframe=DarkGray,
    halign=center,
    sharp corners=south,
    coltext=white,
    #1 % 允许额外选项
}
\newtcolorbox{lightgraybox}[1][]{
    enhanced, % 使用 enhanced 引擎 :cite[10]
    breakable,
    fontupper=\normalsize,
    colback=LightGray,
    colframe=DarkGray,
    sharp corners=north,
    #1 % 允许额外选项
}
\newlength{\DarkGrayWidth}
\newenvironment{graybox}[1][Notation]{
    \settowidth{\DarkGrayWidth}{#1}
    % \needspace{3\baselineskip}
    \begin{darkgraybox}[width=1.25\DarkGrayWidth+3em,after skip=0pt]#1\end{darkgraybox}
    % \vspace*{-3em}
    % \nopagebreak
    % \vspace*{-0.1em}
    \begin{lightgraybox}[before skip=0pt]
}{\end{lightgraybox}}%自定义命令

\begin{document}
\title{Optics}
\author{Wang Yapeng}

\maketitle
\frontmatter
\chapter{preface}
量子力学诞生于19世纪,是物理学的一个重要分支。它主要研究微观粒子的行为和性质,如电子、原子、分子等。量子力学的基本原理包括波粒二象性、不确定性原理、量子叠加态等,这些原理与经典力学有很大的不同。



\mainmatter
\tableofcontents
\chapter{经典统计}
\section{粒子状态的描述}
假设粒子的自由度为$r$,经典力学告诉我们,粒子的运动状态由$r$个广义坐标$q$以及对应的广义动量$p$确定,粒子的能量是其广义坐标和广义动量的函数:
\begin{equation}
    \varepsilon=\varepsilon(\vec{q},\vec{p}).
\end{equation}

$\vec{p}$和$\vec{q}$张成一个$2r$维的线性空间,称为$\mu$空间,粒子任意时刻的状态为$\mu$空间中的一个点,运动轨迹为$\mu$空间中的一条线。

在量子力学中,粒子状态由Hilbert空间中的矢量描述,实际应用中常用一组量子数描述。

下面介绍几个简单例子:

\paragraph{自由粒子}理想气体的分子和金属的自由电子可以视为自由粒子,有三个自由度:
\begin{equation}
    x,y,z,p_x=m\dot{x},p_y=m\dot{y},p_z=m\dot{z}.
\end{equation}
其能量就是粒子的动能
\begin{equation}
    \varepsilon=\frac{1}{2m}\left(p_x^2+p_y^2+p_z^2\right).
\end{equation}

\paragraph{一维谐振子} 一维谐振子只有一个自由度
\begin{equation}
    \varepsilon=\frac{p^2}{2m}+\half m\omega^2.
\end{equation}
其运动轨迹在$\mu$空间中为一个椭圆。

\paragraph{转子} 转子只有两个自由度
\begin{equation}
    \varepsilon=\frac{1}{2I}\left(p_\theta^2+\frac{1}{\sin^2\theta}p_\varphi^2\right)
\end{equation}

在量子力学中转子的能量为
\begin{equation}
    \varepsilon=\frac{M^2}{2I}, M^2=l(l+1)\hbar^2, l=0,\pm1, \pm2, \cdots.
\end{equation}
\begin{equation}
    M_z=m\hbar, m=-l,\cdots,l.
\end{equation}


\section{系统微观状态的描述}
近独立粒子系统是指系统中粒子的相互作用很弱,相互作用势能远小于粒子动能,可以忽略粒子间的相互作用,比如理想气体。

近独立粒子系统中系统的能量可以描述为单粒子能量的和:
\begin{equation}
    E=\sum_i^N \varepsilon_i.
\end{equation}

在经典力学的视角下,全同粒子是可分辨的,这样的系统称为\gls{Boltzman_sysm},而在量子力学视角下,全同粒子是无法分辨的,按照组成粒子的不同分为玻色系统和费米系统。

\begin{theorem}[等概率原理]
    对于平衡状态的孤立系统,系统各个可能的微观状态出现的概率是相等的。
\end{theorem}

对于一个系统中的$N$个粒子,能级$\varepsilon_i$简并度为$\omega_i$,对应$a_i$个粒子,显然系统的总能量为
\begin{equation}
    E=\sum_l a_l\varepsilon_l.
\end{equation}
对于\gls{Boltzman_sysm},系统的微观状态总数为
\begin{equation}
    \Omega_{M.B.}=\frac{N!}{\prod_l a_l!}\prod_l \omega_l^{a_l}.
\end{equation}
对于玻色系统
\begin{equation}
    \Omega_{B.E.}=\prod_l\frac{(\omega+a_l-1)!}{a_l!(\omega_l-1)!}.
\end{equation}
对于费米系统
\begin{equation}
    \Omega_{F.D.}=\prod_l\frac{\omega_l!}{a_l!(\omega_l-a_l)!}.
\end{equation}

当能级中的粒子数远小于该能级的简并度,即$a_l\ll \omega_l$时,后两者都可以近似为
\begin{equation}
    \Omega_{B.E.}\approx\frac{\Omega_{M.B.}}{N!},\quad \Omega_{F.D.}\approx\frac{\Omega_{M.B.}}{N!}.
\end{equation}
也就是说这两个系统的宏观特征与\gls{Boltzman_sysm}是一致的。

\section{Boltzman Distribution}
根据等概率原理,我们可以认为微观状态数最多的分布出现的概率是最大的,因此称为最概然分布。\gls{Boltzman_sysm}粒子的最概然分布就称为\gls{Boltzman_dist}。

在接下来的讨论之前,我们先声明一个近似关系:
\begin{equation}
    \ln m!=m(\ln m-1), m\gg 1.
\end{equation}

\gls{Boltzman_sysm}的微观态总数为
\begin{equation}
    \Omega=\frac{N!}{\prod_l a_l!}\prod_l \omega_l^{a_l}.
\end{equation}
取对数可以得到:
\begin{equation}
    \ln\Omega=N(\ln N-1)-\sum_l a_l(\ln a_l-1)+\sum_l a_l\ln\omega_l.
\end{equation}
对$a_l$分别求导,可以得到微观态总数最多的条件:
\begin{equation}
    a_l=\omega_le^{-\alpha-\beta\varepsilon_l}.
\end{equation}
即处于某一状态的粒子数目与能级简并度成正比,与能级能量的指数成反比。参数$\alpha,\beta$可以通过以下等式计算得到:
\begin{equation}
    N=\sum_s e^{-\alpha-\beta\varepsilon_s}.
\end{equation}
\begin{equation}
    E=\sum_s \varepsilon_s e^{-\alpha-\beta\varepsilon_s}.
\end{equation}

\section{Boltzman统计}
\subsection{热力学量的统计表达}
内能为粒子无规则运动总能量的统计平均:
\begin{equation}
    U=\sum_la_l\varepsilon_l=\sum_l\varepsilon_l\omega_l e^{-\alpha-\beta\varepsilon}.
\end{equation}
引入配分函数$Z_1$:
\begin{equation}
    Z_1=\sum_l\omega_le^{-\beta\varepsilon}.
\end{equation}
则有
\begin{equation}
    N=e^{-\alpha}Z_1.
\end{equation}
\begin{equation}
    U=-N\pp{}{\beta}\ln Z_1.
\end{equation}
即为内能的统计表达式。

压强为
\begin{equation}
    p=\frac{N}{\beta}\pp{}{V}\ln Z_1.
\end{equation}

熵定义为
\begin{equation}
    S=k\ln\Omega
\end{equation}
其中$k=\frac{1}{\beta T}=1.381\times 10^{-23}\mathrm{J\cdot K^{-1}}$,为玻尔兹曼常数。

\section{理想气体状态方程}
理想气体的配分函数为
\begin{equation}
    \begin{aligned}
        Z_1& =\frac{1}{h^3}\iiint\d x\d y\d z \iiint e^{-\beta\frac{p_x^2+p_y^2+p_z^2}{2m}}\d p_x \d p_y \d p_z.\\
        &=V\left(\frac{2\pi m}{h^2\beta}\right)^{3/2}.
    \end{aligned}
\end{equation}

压强为
\begin{equation}
    p=\frac{N}{\beta}\pp{}{V}\ln Z_1=\frac{NkT}{V}.
\end{equation}

\subsection{麦克斯韦分布}
通过玻尔兹曼分布可以得到分子速率的分布
\begin{equation}
    f(v)=A e^{-\frac{mv^2}{2kT}}.
\end{equation}
其中$A=n\left(\frac{m}{2\pi kT}\right)^{3/2}$为归一化常数。这个分布称为麦克斯韦分布。

可以得到粒子的平均速率为\[\bar{v}=\sqrt{\frac{8kT}{\pi m}}\]
平均动能为\[\half m\bar{v^2}=\frac{3}{2}kT.\]

\subsection{能量均分定理}
\begin{theorem}[能量均分定理]
    对于温度为$T$的平衡状态的经典系统,粒子能量中每一个平方项的的平均值(每一个自由度的动能)等于$\half kT$.
\end{theorem}

利用能量均分定理,可以很方便地对系统能量进行求解。例如:
\par 
单原子分子的内能为\[U=\frac{3}{2}NkT.\]
定容热容为\[C_v=\frac{3}{2}kT.\]
定压热容为\[C_p=C_v+kT=\frac{5}{2}kT.\]

双原子分子有5个自由度,分子的内能为\[U=\frac{5}{2}NkT.\]
定容热容为\[C_v=\frac{5}{2}kT.\]
定压热容为\[C_p=C_v+Nk=\frac{7}{2}kT.\]


\chapter{Elctromagnetic Theory}

\chapter{Angular Momentum and Rotation}
\section{Angular Momentum}
定义角动量算符\begin{equation}
    \op{L}=\op{x}\times\op{p}.
\end{equation}

在平面直角坐标中有三个分量:
\begin{align}
    \op{L}_x&=\op{y}\op{p}_z-\op{z}\op{p}_y\\
    \op{L}_y&=\op{z}\op{p}_x-\op{x}\op{p}_z\\
    \op{L}_z&=\op{x}\op{p}_y-\op{y}\op{p}_x\\
    \op{L}^2&=\op{L}_x^2+\op{L}_y^2+\op{L}_z^2
\end{align}

容易证明,角动量算符的对易关系为:
\begin{align}
    \com{\op{L}_x}{\op{L}_x}&=0;\\
    \com{\op{L}_x}{\op{L}_y}&=i\hbar\op{L}_z;\\
    \com{\op{L}_x}{\op{L}_z}&=-i\hbar\op{L}_y.
\end{align}
其余分量可以此类推。

但是角动量的平方与各个分量都是对易的:
\begin{equation}
    \com{\op{L}^2}{\op{L}_i}=0, \quad i=x,y,z.
\end{equation}

由于角动量的三个分量并不对易,这意味着量子系统中只能同时准确观测到总角动量的幅值和某一个方向的角动量分量。我们取$\op{L}^2$和$\op{L}_z$作为描述角动量的一组力学量完全集。我们可以找到一组量子态$\ket{\psi}$,满足:
\begin{equation}
    \op{L}^2\ket{\psi}=\lambda\ket{\psi},\quad \op{L}_z\ket{\psi}=\mu\ket{\psi}.
\end{equation}

\section{the Quantum Number of Angular Momentum}

仿照谐振子中的算符,可以定义角动量的升降算符:
\begin{equation}
    \op{L}_\pm\equiv \op{L}_x\pm\op{L}_y.
\end{equation}

易证,升降算符的对易关系为:
\begin{equation}
    \com{\op{L}^2}{\op{L}_\pm}=0, \quad \com{\op{L}_z}{\op{L}_\pm}=\pm\hbar\op{L}_\pm.
\end{equation}
可以证明,对于$\op{L}^2,\op{L}_z$的共同本征态$\ket{\psi}$,$\op{L}_\pm\ket{\psi}$也是$\op{L}^2,\op{L}_z$的共同本征态.

\begin{graybox}[Proof]
    对于$\op{L}_\pm\ket{\psi}$:

    \begin{equation}
        \begin{aligned}
            \op{L}^2\op{L}_\pm\ket{\psi}=\op{L}_\pm\op{L}^2\ket{\psi}=\lambda\op{L}_\pm\ket{\psi}.
        \end{aligned}
    \end{equation}

    \begin{equation}
        \begin{aligned}
            \op{L}_z\op{L}_\pm\ket{\psi}&=\op{L}_\pm\op{L}_z\ket{\psi}\pm\hbar\op{L}_\pm\ket{\psi}\\
            &=(\mu\pm\hbar)\op{L}_\pm\ket{\psi}.
        \end{aligned}
    \end{equation}
\end{graybox}

与简谐振子相同,$\op{L}_z$的本征值也不可能无限增高或降低,存在两个本征态满足:
\begin{equation}
    \op{L}_+\ket{\psi}_\text{highest}=0,\quad \op{L}_-\ket{\psi}_\text{lowest}=0.
\end{equation}
假设$$\op{L}^z\ket{\psi}_\text{highest} =\lambda_{\max}\ket{\psi}_\text{highest},\quad \op{L}^z\ket{\psi}_\text{lowest} =\lambda_{\min}\ket{\psi}_\text{lowest}$$

可以将$\op{L}^2$写成:
\begin{equation}
    \op{L}^2=\op{L}_\pm\op{L}_\mp+\op{L}_z^2\mp\hbar\op{L}_z.
\end{equation}

那么
\begin{equation}
    \begin{aligned}
        \op{L}^2\ket{\psi}_\text{highest}&=\op{L}_z^2\mp\hbar\op{L}_z\ket{\psi}_\text{highest}\\
        &=\mu^2+\mu\hbar\ket{\psi}_\text{highest}\\
        &=\mu(\mu+\hbar)\ket{\psi}_\text{highest}.
    \end{aligned}
\end{equation}
即$\mu$最大满足$\mu_\max(\mu_\max+\hbar)=\lambda$,同理$\mu$最小满足$\mu_\min(\mu_\min-\hbar)=\lambda$.

这说明$\mu_\min=-\mu_\max$,且$\mu_\max-\mu_\min$为$\hbar$的整数倍。取量子数$m$标记$\op{L}_z$的本征值:
\begin{equation}
    \mu=m\hbar,\quad m=-l,-l+1,\cdots,l-1,l;\quad l=0,1/2,1,3/2,\cdots.
\end{equation}
显然$m$只能为整数或半整数。

对应的\begin{equation}
    \lambda=l(l+1)\hbar^2.
\end{equation}

\section{Rotation}




\chapter{Perturbation Theory}
在量子体系中,薛定谔方程并非总是有解析解的,实际上大部分情况下只有数值解。处理无解析解的量子体系,通常有两种方法:其中在哪个一种方法为路径积分,此处暂且不提,更常用的是微扰论的方法.

在微扰论方法中,可以将系统的哈密顿量是为一个有解析解的哈密顿量加上微扰的形式:
\begin{equation}
    \op{H}=\op{H}_0+\op{H}'
\end{equation}
通过逐阶近似进行求解,对于$\expectket{\op{H}'}\ll\expectket{\op{H}_0}$的情况,往往进行一阶近似或者二阶近似就可以得到相当准确的结果。

\section{Perturbation Theory without Time}
我们先讨论哈密顿量中不显含时间的情况,对于哈密顿量$\op{H}_0$,有对应的本征态和本征值:
\begin{equation}
    \op{H}_0\ket{\psi_{im}}=E_i\ket{\psi_{im}}.
\end{equation}
$m$代表哈密顿量的本征态可能存在简并。

假设总哈密顿量对应的本征态和本征值可以写为:
\begin{equation}
    \ket{\phi_{im}}=\ket{\phi_{im}^0}+\ket{\phi_{im}^1}+\ket{\phi_{im}^2}+\cdots.
\end{equation}
\begin{equation}
    E=E^0+E^1+E^2+\cdots.
\end{equation}

借鉴线性代数中的系数对比方法,我们规定本征态中各阶项都是正交的,将其带入方程
\begin{equation}
    \left(\op{H}_0+\op{H}'\right)\ket{\phi_{im}}=E\ket{\phi_{im}}
\end{equation}
并对比不同阶的项可以得到:
\begin{equation}
    \label{Pertur_orders}
    \begin{aligned}
        \op{H}_0\ket{\phi^0}&=E^0\ket{\phi^0}\\
        \op{H}_0\ket{\phi^1}+\op{H}'\ket{\phi^0}&=E^1\ket{\phi^0}+E^0\ket{\phi^1}\\
        \op{H}_0\ket{\phi^2}+\op{H}'\ket{\phi^1}&=E^2\ket{\phi^0}+E^1\ket{\phi^1}+E^0\ket{\phi^2}\\
        \cdots&=\cdots
    \end{aligned}
\end{equation}

显然$\ket{\phi_0},E^0$是$\op{H}_0$的本征态和对应本征值,式\ref{Pertur_orders}中,各式左乘$\bra{\phi^0}$可以得到能阿玲本征值的高阶项:
\begin{equation}
    \begin{aligned}
    E^1&=\bra{\phi^0}\op{H}'\ket{\phi^0}\\
    E^2&=\bra{\phi^0}\op{H}'\ket{\phi^1}\\
    \cdots&=\cdots
    \end{aligned}
\end{equation}

左乘$\op{H}_0$的本征态$\ket{\psi_{im}}$可以得到:
\begin{equation}
    \begin{aligned}
        (E_i-E^0)\braket{\psi_{im}}{\phi^1}&=-\bra{\psi_{im}}\op{H}'\ket{\phi^0}.\\
        (E_i-E^0)\braket{\psi_{im}}{\phi^2}&=E^1\braket{\psi_{im}}{\phi^1}-\bra{\psi_{im}}\op{H}'\ket{\phi^1}.
    \end{aligned}
\end{equation}

再通过
\begin{equation}
    \ket{\phi^l}=\sum_{im}\ket{\psi_{im}}\braket{\psi_{im}}{\phi^l}
\end{equation}
即可得到本征态的高阶近似。

但是对于有简并的情况$E_i-E_0=0$,不能通过这种方法直接求解,也无法直接得到$\ket{\phi^0}$.一般地,将$\ket{\phi^0}$视为多个简并态的叠加:
\begin{equation}
    \ket{\phi^0}=\sum_{m}\ket{\psi_{im}}\braket{\psi_{im}}{\phi^0}.
\end{equation}
代入可以得到
\begin{equation}
    E^1\braket{\psi_{im}}{\phi^0}=\bra{\psi_{im}}\op{H}'\ket{\phi^0}.
\end{equation}
通常将其写成矩阵的形式:
\begin{equation}
    \sum_m\left(H'_{m'm}-E^1\delta_{m'm}\right)\braket{\psi_{im}}{\phi^0}=0
\end{equation}
等价于
\begin{equation}
    \det\left(H'-E^1I\right)=0.
\end{equation}
即$E^1$是$\op{H}'$在矩阵表象下的本征值。

\section{Scatterring Problem}



\chapter{Fourier Optics}

\chapter{Laser}
\appendix
\chapter{Mathematical Equations}
\section{Hermitian ploynomial}
\section{Sphere Harmonic Function}
\printglossaries
\addcontentsline{toc}{chapter}{Glossaries}

\backmatter 
% \printindex
\end{document}