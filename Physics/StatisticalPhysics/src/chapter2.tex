\chapter{量子统计}
\section{粒子的量子描述}
\section{玻色分布和费米分布}
对于玻色系统
\begin{equation}
    \Omega=\prod_l\frac{(\omega_l+a_l-1)}{a_l!(\omega-1)!}
\end{equation}
对应的最概然分布为:
\begin{equation}
    a_l=\frac{\omega_l}{e^{\alpha+\beta\varepsilon_l}-1}.
\end{equation}
称为玻色爱因斯坦分布。


对于费米系统 
\begin{equation}
    \prod_l\frac{\omega_l!}{a_l!(\omega_l-a_l)!}
\end{equation}
对应的最该然分布为
\begin{equation}
    a_l=\frac{\omega_l}{e^{\alpha+\beta\varepsilon_l}+1}.
\end{equation}
称为费米狄拉克分布。

当$a_l\ll \omega_l$时,二者都等价于玻尔兹曼分布,即经典情形。

\section{玻色系统和费米系统中热力学量的表达式}
\subsection{玻色系统}
对于玻色系统,引入巨配分函数:
\begin{equation}
    \Xi=\prod_l\Xi_l=\prod_l\left(1-e^{-\alpha-\beta\varepsilon_l}\right)^{-\omega_l}.
\end{equation}
取对数为:
\begin{equation}
    \ln\Xi=-\sum_l\omega_l\ln(1-e^{-\alpha-\beta\varepsilon_l}).
\end{equation}
系统总粒子数为
\begin{equation}
    N=-\pp{}{\alpha}\ln\Xi.
\end{equation}
系统内能为:
\begin{equation}
    U=-\pp{}{\beta}\ln \Xi.
\end{equation}
压强为:
\begin{equation}
    p=\frac{1}{\beta}\pp{}{V}\ln\Xi.
\end{equation}
由于
\begin{equation}
    \d S=\frac{1}{T}\left(\d U-p\d V -\mu\d\bar{N}\right).
\end{equation}
可以得到
\begin{equation}
    \beta=\frac{1}{kT}, \alpha =-\frac{\mu}{kT}.
\end{equation}
积分可以得到:
\begin{equation}
    S=k\ln\Omega
\end{equation}
即玻尔兹曼关系。

\subsection{费米系统}
费米系统的巨配分函数为:
\begin{equation}
    \Xi=\prod_l\left(1+e^{-\alpha-\beta\varepsilon_l}\right)^{\omega_l}.
\end{equation}
其余与玻色系统类似。

\section{量子系统的去经典化}
在体积$V$内,能量范围$[\varepsilon,\varepsilon+\d \varepsilon]$内,分子可能为微观状态数目为:
\begin{equation}
D(\varepsilon)\d \varepsilon=g \frac{V}{4\pi^2\hbar^3\varepsilon}\left(2m\varepsilon\right)^{3/2}\d\varepsilon.    
\end{equation}
其中$g$为可能存在自旋而引入的简并度。

\section{玻色爱因斯坦凝聚}
\section{.}
\chapter{系综理论}