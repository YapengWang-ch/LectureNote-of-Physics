\renewcommand{\figurename}{图}
\renewcommand{\tablename}{表}
\renewcommand{\glossaryname}{术语表}
%%三线表线距控制
\def\topline{\hline\hline\specialrule{0pt}{3pt}{0pt}}%双线,下间距2pt
\def\midline{\specialrule{0pt}{3pt}{0pt}\hline\specialrule{0pt}{3pt}{0pt}}%单线,下间距2pt	
\def\bottomline{\specialrule{0pt}{2pt}{0pt}\hline\hline}%双线,上间距2pt
%\renewcommand\arraystretch{1.2} %控制表格行距,置于table环境内
\renewcommand{\abstractname}{摘要}
\newcommand*{\num}{pi}

\newtheorem{theorem}{Theorem}[chapter]
\newtheorem{definition}{Definition}[chapter]
\newtheorem{lem}[theorem]{Lemma}

\newcommand{\op}[1]{\hat{#1}}%算符
\newcommand{\ket}[1]{\left|#1\right>}
\newcommand{\bra}[1]{\left<#1\right|}
\newcommand{\braket}[2]{\left<#1\middle|#2\right>}
\newcommand{\com}[2]{\left[#1,#2\right]}%对易子
\newcommand{\anticom}[2]{\left\lbrace#1,#2\right\rbrace}%反对易子
\newcommand{\expect}[1]{\overline{#1}}%期望,上划线形式
\newcommand{\expectket}[1]{\left<#1\right>}%算符的期望
\newcommand{\ampsq}[1]{\left|#1\right|^2}%幅值平方,复数平方
\newcommand{\half}{\frac{1}{2}}
\newcommand{\quat}{\frac{1}{4}}
\newcommand{\si}[1]{\,\mathrm{#1}} %单位格式
\newcommand{\der}{\mathrm{d}}				%导数符号
\newcommand{\dd}[2]{\frac{\der #1}{\der #2}}
\newcommand{\ddto}[2]{\frac{\der^2 #1}{\der #2^2}}
\newcommand{\dx}[1]{\dd{#1}{x}}
\newcommand{\dt}[1]{\dd{#1}{t}}
\newcommand{\p}{\partial}				%偏微分符号
\newcommand{\pp}[2]{\frac{\p #1}{\p #2}}
\newcommand{\ppto}[2]{\frac{\p^2 #1}{\p #2^2}}
\newcommand{\px}[1]{\pp{#1}{x}}
\newcommand{\pt}[1]{\pp{#1}{t}}
\renewcommand{\d}{\der}
\newcommand{\degree}{^\circ}

\renewcommand{\max}{\text{max}}
\renewcommand{\min}{\text{min}}

\tcbuselibrary{skins, breakable}
\definecolor{DarkGray}{RGB}{70,70,70}
\definecolor{LightGray}{RGB}{240,240,240}
\newtcolorbox{darkgraybox}[1][]{
    enhanced, % 使用 enhanced 引擎 :cite[10]
    fontupper=\large,
    colback=DarkGray,
    colframe=DarkGray,
    halign=center,
    sharp corners=south,
    coltext=white,
    #1 % 允许额外选项
}
\newtcolorbox{lightgraybox}[1][]{
    enhanced, % 使用 enhanced 引擎 :cite[10]
    breakable,
    fontupper=\normalsize,
    colback=LightGray,
    colframe=DarkGray,
    sharp corners=north,
    #1 % 允许额外选项
}
\newlength{\DarkGrayWidth}
\newenvironment{graybox}[1][Notation]{
    \settowidth{\DarkGrayWidth}{#1}
    \begin{darkgraybox}[width=1.25\DarkGrayWidth+3em,after skip=0pt]#1\end{darkgraybox}
    % \vspace*{-3em}
    \begin{lightgraybox}[before skip=0pt]
}{\end{lightgraybox}}